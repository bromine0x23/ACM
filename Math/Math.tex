\SetPath{Math}

\chapter{数学相关}

\section{数论算法}
\subsection{欧几里德(Euclid)算法}
\CPPSource{euclid}
\RubySource{euclid}
\subsection{快速幂}
\CPPSource{fast_pow}
\subsection{素数筛法}
\CPPSource{prime_sieve}
\subsection{欧拉函数}
欧拉函数$\phi(n)$的值为不大于n的正整数中与n互质的数的个数\\
若
\begin{displaymath}
n = p_1^{k_1} p_2^{k_2} \cdots p_r^{k_r}
\end{displaymath}
则
\begin{displaymath}
\phi(n) = \prod_{i=1}^{r} p_{i}^{k_{i}-1}(p_{i}-1) = n\prod_{p|n}\left(1-\frac{1}{p}\right)
\end{displaymath}\\
\begin{displaymath}
\;\phi(n^{m}) = n^{m-1}\phi(n) 
\end{displaymath}
\CPPSource{Euler}
\subsection{欧拉定理}
若$n$、$a$为正整数,且$n$、$a$互质(即$\gcd(n,a)=1$),则
\begin{displaymath}
a^{\phi(n)} \equiv 1 \pmod n
\end{displaymath}
\subsection{费马小定理}
如果$p$是素数,则
\begin{displaymath}
a^{p} \equiv a \pmod{p}
\end{displaymath}
或在$a$与$p$互质时
\begin{displaymath}
a^{p-1} \equiv  1 \pmod{p}
\end{displaymath}
\subsection{中国剩余定理}
\CPPSource{ChineseRem}
\RubySource{ChineseRem}
\subsection{Miller Rabin算法(素性测试)}
\CPPSource{miller_rabin}
\subsection{Pollard Brent算法(因数分解)}
\CPPSource{pollard_brent}

\section{组合数学}
\subsection{组合数(二项式系数)}
从$n$个元素中选出$k$个元素的方法数,记为$C_{n}^{k}$或$C(n, k)$
\begin{displaymath}
C_n^k = \frac{n!}{k!\;(n-k)!} \quad
\end{displaymath}
\begin{displaymath}
C_{n}^{k} = 
\begin{cases}
C_{n-1}^{k-1} + C_{n-1}^{k} & 1 \leqslant k \leqslant n\\
1 & k=0
\end{cases}
\end{displaymath}
$k$元子集包含第$n$个元素时,方法数为$ C_{n-1}^{k-1}$;不包含时,方法数为$C_{n-1}^{k}$\\
\CPPSource{combine}

\subsection{Fibonacci数}
\begin{displaymath}
F_{n} = \frac{1}{\sqrt{5}}\left\{\left(\frac{1+\sqrt{5}}{2}\right)^n - \left(\frac{1-\sqrt{5}}{2}\right)^n\right\}
\end{displaymath}
\begin{displaymath}
F_{n} = 
\begin{cases}
F_{n-1}+ F_{n-2} & n \geqslant 2\\
1 & n=1\\
0 & n=0
\end{cases}
\end{displaymath}
\begin{displaymath}
\begin{pmatrix} F_{n+2} & F_{n+1} \\ F_{n+1} & F_{n} \end{pmatrix}
=
\begin{pmatrix} 1 & 1 \\ 1 & 0 \end{pmatrix}^{n + 1}
\end{displaymath}
\CPPSource{fibonacci}

\subsection{第一类Stirling数}
有正负,其绝对值是$n$个元素的项目分作$k$个循环排列($n$个人分成$k$组,组内按特定顺序围圈)的方法数,记为$s(n,k)$
\begin{displaymath}
x(x-1)(x-2)\cdots(x-n+1) = \sum_{k=1}^{n} s(n, k) x^{k}
\end{displaymath}
\begin{displaymath}
|s(n, k)| = 
\begin{cases}
|s(n-1, k-1)| + (n-1)\;|s(n-1, k)| & 2 \leqslant k \leqslant n\\
1 & n=k=1
\end{cases}
\end{displaymath}
第$n$个元素单独够成一个循环排列,前$n-1$个元素构成$k-1$个循环排列,方法数为$s(n-1, k-1)$;前$n-1$个元素构成$k$个循环排列,第$n$个元素某一元素的左边,方法数为$(n-1)\;s(n-1, k)$\\
\CPPSource{stirling1}

\subsection{第二类Stirling数}
$n$个元素的集定义$k$个等价类(亦即$n$个人分成$k$组)的方法数,记为$S(n,k)$
\begin{displaymath}
S(n, k) = \frac{1}{k!}\sum_{j=1}^{k} (-1)^{k-j} C_{k}^{j} j^{n}
\end{displaymath}
\begin{displaymath}
S(n, k) = 
\begin{cases}
S(n-1, k-1) + k\;S(n-1, k) & 2 \leqslant k \leqslant n\\
1 & n=k=1
\end{cases}
\end{displaymath}
第$n$个元素单独够成一个等价类,前$n-1$个元素构成$k-1$个等价类,方法数为$S(n-1, k-1)$;前$n-1$个元素构成$k$个等价类,第$n$个元素放入其中一个,方法数为$k\;S(n-1, k)$\\
\CPPSource{stirling2}

\subsection{Catalan数}
$n$个节点组成的不同构二叉树个数\\
$2n+1$个节点组成的不同构满二叉树数
\begin{displaymath}
C_{n} = \frac{1}{n+1} C_{2n}^{n} = \frac{(2n)!}{(n+1)!\;n!}
\end{displaymath}
\begin{displaymath}
C_{n} = C_{2n}^{n} - C_{2n}^{n+1} \quad\mbox{for } n\geqslant 1
\end{displaymath}
\begin{displaymath}
C_{0} = 1 \quad\mbox{and}\quad C_{n+1} = \sum_{i=0}^{n} C_{i}\;C_{n-i} \quad\mbox{for } n\geqslant 0
\end{displaymath}
\begin{displaymath}
C_{0} = 1 \quad\mbox{and}\quad C_{n} = \frac{2(2n-1)}{n+1} C_{n-1} \quad\mbox{for } n>0
\end{displaymath}\\
\CPPSource{catalan}

\subsection{Bell数}
基数为$n$的集合的划分数
\begin{displaymath}
B_{n+1} = \sum_{k=0}^{n} C_{n}^{k}\;B_{k}
\end{displaymath}
\begin{displaymath}
B_{p+n} \equiv B_{n} + B_{n+1} \pmod p
\end{displaymath}
\begin{displaymath}
B_{n} = \sum_{k=1}^{n} S(n, k)
\end{displaymath}\\
\CPPSource{bell}

\subsection{康托(Cantor)展开}
\CPPSource{Cantor}

\subsection{卢卡斯(Lucas)定理}
\CPPSource{Lucas}

\section{大数运算}
\CPPSource{bignum}

\section{矩阵运算}
\CPPSource{matrix}